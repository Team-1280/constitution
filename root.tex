\documentclass{constitution}

\author{Youwen Wu}
\sponsor{}

\first{Warren Lin}
\second{}
\third{}
\captain{Matthew Fletcher}

\aye{}
\nay{}
\abstain{}
\status{IN CONSIDERATION}

\date{\today}

\billtitle{Team 1280 Constitution}

\billcode{0.1.0}

\begin{document}
\maketitle

% Welcome to Team 1280's constitution source code. This document is being written
% publicly, and you are free to suggest and contribute changes or improvements.

\center{\footnotesize{Everyone is permitted to copy and distribute verbatim copies of this document, but changing it is not allowed.}}
\vspace{12pt}
\linebreak
\small{The key words “MUST”, “MUST NOT”, “REQUIRED”, “SHALL”, “SHALL NOT”, “SHOULD”, “SHOULD NOT”, “RECOMMENDED”, “MAY”, and “OPTIONAL” in this document are to be interpreted as described in \href{https://datatracker.ietf.org/doc/html/rfc2119}{RFC 2119.}}

\center{\textbf{Preamble}}

\begin{adjustwidth}{80pt}{80pt}
	\normalsize{
		This constitution is Team 1280's official governing text, and it outlines the rules and regulations that all members must adhere to.
		The constitutions for most robotics teams and other organizations are designed to take away your freedom to innovate, revolutionize, and change, in favor of nebulous goals. By contrast, the Team 1280 Constitution is intended to guarantee your freedom within the club and team--to make sure it remains free for all its members.
		When we refer to the term `free' in this document, we refer to free as in `free speech', not `free beer'.
		We, the members of Team 1280, in order to promote the development of robotics and STEM education in the SRVHS community, hereby establish this constitution to define our standard procedures, safeguards, and ensure the freedom of all members.
	}
\end{adjustwidth}

\normalsize
\vspace{12pt}
\vspace{12pt}

\whereas{Team 1280 seeks to promote the development of robotics and STEM education in the SRVHS community}

\whereas{There is a need to ensure fairness, respect, and transparency among members and leadership alike}

\whereas{Team 1280 has, not only the ability, but also the obligation, to guarantee certain rights for all of its members}

\whereas{Such an undertaking must also require the collaboration of several individuals on campus}

\clearpage

\therefore

\article{1}{Membership in Team 1280 is open to all students currently enrolled at San Ramon Valley High School, regardless of grade level, experience, or background.}
\begin{sub}
	\item Members are expected to attend meetings, participate in team activities, and contribute to the team in a positive manner.
	\item Members are expected to follow all school policies and the rules set forth in this constitution.
	\item Members are expected to treat all other members with respect and dignity.
	\item Members are expected to follow the directions of the leadership and advisors.
\end{sub}

\article{2}{Leadership and Governance
	\begin{sub}
		\item The leadership shall be established by processes outlined in Article VI.
		\item The leadership shall be responsible for overseeing day-to-day operations, setting team goals and priorities, managing the budget, and ensuring compliance with school policies and this constitution. Leadership shall make decisions in a transparent manner with input from the general membership.
		\item Leadership meetings shall be open to all members to attend as observers. Minutes from each meeting documenting decisions and discussions shall be shared with the full team in a timely manner. Leadership shall hold open forums at least once per quarter to solicit feedback and input from members.
		\item The leadership must make all of its major decisions in the purview of all members; this includes, but is not limited to, the selection of competition events, the selection of team officers, and the allocation of funds.
		\item At its discretion, leadership may make minor decisions without consulting the full team. However, all decisions must be communicated to the full team in a timely manner.
		\item In certain emergency situations, it is understood that it may not be feasible for major decisions to be made with the full team. In these cases, leadership may make decisions as necessary, but must consult the full team as soon as possible. Any lasting decisions must be ratified by the full team in a vote. Additionally, if abuse of this power is suspected, the full team may call for a vote of no confidence in the leadership.
	\end{sub}
}

\article{3}{Team 1280 shall hold regular meetings at least once per week during the build season. Meetings shall be held at a time and place that is convenient for the majority of members.}

\article{4}{Team 1280 does not endorse the use of any proprietary software. Whenever possible, the team shall use \textit{free and open source} software and hardware.
	\begin{sub}
		\item The use of proprietary software in a team that claims to promote the proliferation and accessibility of STEM education is hypocritical and unacceptable. Therefore, we establish the following guidelines.
		\item The team shall commit to using free and open source solutions whenever possible over proprietary (closed source or restrictively-licensed solutions). This includes software whose source code is both made available for public use and can be modified or redistributed by the general public.
		\item The team shall refuse to use any form of SaaSS, or ``Service as a Software Substitute''. This includes, but is not limited to, Google Docs, Microsoft Office 365, Adobe Creative Cloud, and onShape. SaaSS is any \textit{service} which provides the opportunity for computing to be done on a \textit{remote server}, which the team does not own. To stay committed to its principles, Team 1280 shall do all of its computing on its own hardware.
	\end{sub}
}

\article{5}{The team shall establish various ``working groups'' to manage the different aspects of the team.
	\begin{sub}
		\item The business and general affairs working group shall be established to manage the team's finances, outreach, and other non-technical affairs.
		\item The mechanical and computer-aided design working group shall be established to manage the team's mechanical design, fabrication, and CAD.
		\item The controls working group shall be established to oversee the electrical and programming operations of the team, and build cohesion between the two distinct fields to establish an overarching ``controls'' system.
		\item The working groups should not be exclusive; members may participate in multiple working groups if they wish, and should not be discouraged from doing so. However, members are encouraged to focus on one working group when necessary to maximize their impact and utility.
	\end{sub}
}

\article{6}{Leadership positions should be established within each working group to allow for better organization and delegation of tasks.
	\begin{sub}
		\item Leadership positions shall be decided by a bureaucratic council of graduating seniors prior to the start of each school year.
		\item Leadership positions shall be open to all members of the team, regardless of grade level.
		\item Leadership should have ultimate control over the technical direction of their respective working groups
	\end{sub}
}

\article{7}{Request For Comments (RFCs) may be drafted by any member of the team to propose changes and goals within the working groups.
	\begin{sub}
		\item RFCs shall be thoroughly reviewed by leadership and made available to the general membership to come to a consensus.
		\item It is recommended that leadership generally defer to the results of RFCs, as they represent the most democratic decisions possible. However, they have ultimate discretion, and may direct their working group as they see fit.
		\item RFCs may be proposed to make modifications to leadership.
	\end{sub}
}

\article{9}{Any ``vote'' held for any purposes described in this constitution are subject to the following guidelines.
	\begin{sub}
		\item A vote must be held at a general meeting, and all members must be notified at least one week in advance. However, if not all team members are in attending, then a suitable alternative should be provided for members to vote remotely.
		\item If not all team members participate in a vote, then votes will be tallied out of the total amount of members who voted. For example, if only $75\%$ of the team participates in a vote, and the vote requires a majority to pass, then only a majority of those $75\%$ is required to pass the vote.
		\item However, votes must be made as accessible as possible, according to the guidelines outlined earlier, to ensure anyone who would like to participate is able to. If the team finds substantial evidence that a vote was held in bad faith with the intention of excluding members, then the vote may be nullified.
		\item Votes shall be conducted by a secret ballot, and the results shall be recorded in the minutes of the meeting.
		\item Votes may be held by either a plurality or a majority. Some articles may specify. Otherwise, it's up to a the discretion of the leadership.
		\item In the event of a tie, the captain shall have the deciding vote.
	\end{sub}
}

\article{8}{Amending or replacing this constitution requires the approval of a \textbf{majority} (not just at \textit{plurality}) of the general membership.
	\begin{sub}
		\item Amendments may be proposed by any member of the team.
		\item Amendments shall be reviewed by leadership and made available to the general membership to come to a consensus.
		\item Amendments shall be proposed by a process similar to those for RFCs.
		\item Whenever this document is updated, increment its version number (located at the very top) according to the Semanntic Versioning v2.0.0 standard. Given the current version number vMAJOR.MINOR.PATCH, and given a major change in contradiction with previous articles, increment the MAJOR number, a minor change that adds some minor enhancements or rules, increment the MINOR number, and a change that corrects spelling mistakes or grammatical errors, increment the PATCH number.
	\end{sub}
}

\article{8}{This constitution shall be ratified upon the approval of $\frac{3}{4}$ of the bureaucratic senior council, or upon a successful vote to ratify by a plurality of the general membership, whichever comes first.}


\signatures
\end{document}
